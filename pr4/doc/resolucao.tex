\section{Lista Sequencial Estática}
\subsection{}
Implementado. 

\subsection{}
Operação Procura
\begin{lstlisting}[language=C]
	int Procura(Lista *li, int x){
		for(int i = 0; i<= li->qtd; i++){
			if(li->dados[i] == x) return i;
		}
		
		return -1;
	}
\end{lstlisting}

\subsection{} Inserção ordenada. 
\begin{lstlisting}[language=C]
int novoInsere(Lista* li, int elem){
	if(li == NULL) return 0;
	if(!listaCheia(li)){
		li->dados[li->qtd] = elem;
		li->qtd++;
		
		qsort(li->dados, li->qtd, sizeof(int), comparador);
		return 1;
	}
	else return 0;
}
\end{lstlisting}

\newpage
\subsection{} Remoção de um elemento $x \in \ Lista$

\begin{lstlisting}[language=C]
	void removeLi(Lista *li, int item) {
		for(int i = 0; i < li->qtd; i++){
			if(li->dados[i] == item){
				if (i < li->qtd - 1){
					for (int j = i; j < li->qtd - 1; j++){
						li->dados[j] = li->dados[j + 1];
					}
				}
				li->qtd--;
				return;
			}
		}
	}
\end{lstlisting}

\section{Lista Simplesmente Encadeada}

\subsection{} 
O TAD foi reimplementado. O desenho rastreio não foi incluído já que para este autor ele é uma instância pertencente ao reino das ideias, na analogia platônica.  
\label{sec:21} 

\subsection{} Tamanho e Procura
\begin{lstlisting}[language=C]
	int tamanhoLi(Lista *li){
		struct NO *corrente = (*li)->prox;
		if(corrente == NULL) return 0;
		
		int tam = 1;
		while(corrente != NULL){
			tam++;
			corrente = corrente->prox;
		}
		return tam;
	}
	
	int procuraLi(Lista *li, int x){
		struct NO *corrente = (*li)->prox;
		if(corrente == NULL) return -1;
		
		int i = 1;
		while(corrente != NULL){
			if(corrente->info == x) return i;
			corrente = corrente->prox;
			i++;
		}
		return -1;
	}
\end{lstlisting}


\subsection{} Inserção ordenada

\begin{lstlisting}
	void insereOrdenado(Lista* lista, int valor){
		NO* novo = alocarNO();
		
		novo->info = valor;
		novo->prox = NULL;
		
		NO* atual = *lista;
		NO* anterior = NULL;
		
		while(atual != NULL && atual->info < valor){
			anterior = atual;
			atual = atual->prox;
		}
		
		if(anterior == NULL){
			novo->prox = *lista;
			*lista = novo;
		} 
		
		else{
			anterior->prox = novo;
			novo->prox = atual;
		}
	}
\end{lstlisting}

\subsection{} Remoção de um elemento. 
\begin{lstlisting}
	void removerElemento(Lista* lista, int valor) {
		NO* atual = *lista;
		NO* anterior = NULL;
		
		while(atual != NULL && atual->info != valor){
			anterior = atual;
			atual = atual->prox;
		}
		
		if(atual == NULL) return;
		
		if(anterior == NULL) *lista = atual->prox;
		else anterior->prox = atual->prox;
		
		free(atual);
	}
\end{lstlisting}

\section{Lista Duplamente Encadeada}
\subsection{} Mesma  situação da seção \ref{sec:21}. 
\subsection{} Tamanho e Procura

\begin{lstlisting}[language=C]
	int tamanho(Lista* lista){
		int tamanho = 0;
		NO* atual = *lista;
		
		while(atual != NULL){
			tamanho++;
			atual = atual->prox; 
		}
		
		return tamanho;
	}
	
	int procura(Lista* lista, int x){
		NO* atual = *lista; 
		
		while(atual != NULL){
			if (atual->info == x){
				return 1; 
			}
			atual = atual->prox; 
		}
		
		return 0; 
	}
\end{lstlisting}

\subsection{} Inserção ordenada
\begin{lstlisting}[language=C]
	void inserirOrdenado(Lista* lista, int x){
		NO* novoNo = alocarNO();
		novoNo->info = x; 
		
		if(*lista == NULL){
			novoNo->prox = NULL;
			novoNo->ant = NULL;
			*lista = novoNo;
		} 
		else if(x <= (*lista)->info){
			novoNo->prox = *lista;
			novoNo->ant = NULL;
			(*lista)->ant = novoNo;
			*lista = novoNo;
		} 
		else{
			NO* atual = *lista;
			
			while(atual->prox != NULL && atual->prox->info < x){
				atual = atual->prox;
			}
			
			novoNo->prox = atual->prox;
			novoNo->ant = atual;
			
			if(atual->prox != NULL){
				atual->prox->ant = novoNo;
			}
			
			atual->prox = novoNo;
		}
	}
\end{lstlisting}

\subsection{} Remoção

\begin{lstlisting}[language=C]
	void removerElemento(Lista* lista, int valor) {
		if (*lista == NULL) return;
		
		NO* atual = *lista;
		
		if(atual->info == valor){
			*lista = atual->prox; 
			if(*lista != NULL){
				(*lista)->ant = NULL;
			}
			free(atual); 
			return;
		}
		
		while(atual != NULL && atual->info != valor){
			atual = atual->prox;
		}
		
		if(atual == NULL) return;
		if(atual->ant != NULL) atual->ant->prox = atual->prox;
		if(atual->prox != NULL) atual->prox->ant = atual->ant;
		
		free(atual); 
	}
\end{lstlisting}

\section{Lista Circular Simplesmente Encadeada}
\subsection{} Desenho mental. 
\subsection{} Tamanho e Procura.
\begin{lstlisting}[language=C]
	int tamanho(Lista* li){
		if(li == NULL || *li == NULL) return 0;
		
		int tamanho = 0;
		NO* aux = *li;
		
		do{
			tamanho++;
			aux = aux->prox;
		}while (aux != *li);
		
		return tamanho;
	}
	
	int procura(Lista* li, int elem){
		if(li == NULL || *li == NULL) return -1;
		
		int posicao = 0;
		NO* aux = *li;
		
		do{
			if(aux->info == elem) return posicao;
			
			posicao++;
			aux = aux->prox;
		}while(aux != *li);
		
		return -1; 
	}
\end{lstlisting}

\subsection{Código}

\href{https://github.com/RodrigoZonzin/labaeds2}{https://github.com/RodrigoZonzin/labaeds2}